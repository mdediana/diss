%% ------------------------------------------------------------------------- %%
\chapter{Conclus�es} \label{cap:conclusoes}

% Os resultados dos exp 2k s�o mais interessantes que o resultado final

% Esse trabalho oferece uma infraestrutura para outros extudo do g�nero
% (considerando disco, por ex), em especial se o Riak for utilizado, dado que
% muito da automatiza��o (start/stop, m�tricas, logs, \emph{benchmark}, etc.) �
% espec�fica dele.

%% ------------------------------------------------------------------------- %%
\section{Contribui��es} \label{sec:contribuicoes}

%% ------------------------------------------------------------------------- %%
\section{Li��es Aprendidas} \label{sec:licoes_aprendidas}

% A arte da an�lise de desempenho: muitas atividade da an�lise de desempenho n�o
% s�o facilmente trat�veis metodologicamente, o trabalho acaba precisando de
% muito empirismo.

% Reprodutibilidade

%Import�ncia de ferramentas p/ pesquisa na �rea de sistemas: linguagens usadas,
%linux (cache, rede, disco), ferramentas de so (top, vmstat, iostat), etc, at�
%github (footnote: este texto est� no gh)

%% ------------------------------------------------------------------------- %%
\section{Produ��es ao Longo do Mestrado}
\label{sec:producoes_ao_longo_do_mestrado}

% semin�rios ime apr tdc est�gios (?) artigo do wtdbd c�digo do riak
% infraestrutura de experimentos no g5k

%% ------------------------------------------------------------------------- %%
\section{Trabalhos Futuros} \label{sec:trabalhos_futuros}

% discos diferentes outros modelos de consist�ncia outras opera��es
% aws
